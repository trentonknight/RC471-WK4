% article example for classicthesis.sty
\documentclass[11pt,a4paper]{scrartcl} % KOMA-Script article 
\usepackage[round]{natbib}
\usepackage{lipsum}
\usepackage{url}
\usepackage[LabelsAligned]{currvita} % nice cv style
\usepackage[nochapters]{classicthesis} % nochapters
\usepackage{tikz}
\usepackage{amsthm}
\usepackage{setspace}
\usepackage{bibleref}
\usetikzlibrary{calc,shapes,arrows,automata,trees,shadows,decorations.pathmorphing,positioning,
shapes.misc,shapes.arrows,chains,matrix,scopes,decorations.pathmorphing,backgrounds}
\renewcommand*{\cvheadingfont}{\LARGE\color{OliveGreen}}
\renewcommand*{\cvlistheadingfont}{\large}
\renewcommand*{\cvlabelfont}{\qquad}
\begin{document}
\pagecolor{Gray!20!Bittersweet!10}
%Coverletter
\begin{cv}{\spacedallcaps{Billabong or River}}
        \begin{cvlist}{\textcolor{Sepia}{\spacedlowsmallcaps{Jason~N~Mansfield}}}\label{PersDat}  
            \item   Regis University
            \item   3333\\
                    Regis Boulevard Denver \\	
                    Colorado 80221-1099
            \item   mansf843@regis.edu\\				
                    \url{http://www.regis.edu/}				
        \end{cvlist}
        \begin{cvlist}{\spacedlowsmallcaps{RC~471}}\label{irgendwas}
            \item Instructed by Professor~Henri~Tshibambe\\
             \url{http://tinyurl.com/3htorkr}
        \end{cvlist}
    \end{cv}
\clearpage

\noindent
\textcolor{OliveGreen}{\spacedallcaps{Egil's Saga}}\\
\textcolor{Sepia}{\spacedlowsmallcaps{The slaying of Bergonund and Rognvald the king's son.}\\
\citet[chapter 60]{sagadb}}\\
\begin{verse}
'Forest-foe, fiercely blowing,\\
Flogs hard and unceasing\\
With sharp storm the sea-way\\
That ship's stern doth plow.\\
The wind, willow-render,\\
With icy gust ruthless\\
Our sea-swan doth buffet\\
O'er bowsprit and beak.'\\
\end{verse}
\clearpage
%Title
\title{\textcolor{OliveGreen}{\rmfamily\normalfont\spacedallcaps{Billabong or River}}}
    \author{\textcolor{Sepia}{\spacedlowsmallcaps{Jason N Mansfield}}}
    \date{} % no date
    
    \maketitle
    
    \begin{abstract}
Throughout life I have experienced times where my spirituality seemed like a torrent. I have also experienced times where I felt like I was cut off from the the Headwaters and forming a Billabong, or worse, a dry river bed. I am still learning how to seek God and my own spirituality. Above concerns for myself, I am a father now and have children I must help guide in this river of life. This notion makes my search all the more critical.
    \end{abstract}
       
    \tableofcontents
    
    \section{Headwaters}
\begin{doublespace}
In the beginning of my life I had two parents who valued their relationship with God and put him first. I was able to read at a very early age due to the Bible studies and associated reading. My brother, sister and I would study from the bible at least two times a week if not more. This was not including the attendance of what the Jehovah's Witness call the Kingdom Hall for additional reading and studies. I can honestly say that my parents did everything they could to ensure I understood the importance of God's word. Acknowledging this, I would have to say my parents were a huge factor in my spiritual growth. Up until I was a young adult I did not stray from the faith I was raised in. Specifically, I remained a Jehovah's Witness and could not imagine living any other way. After a series of family tragedies and life changes I felt isolated from my original teachings and lifestyle of the Jehovah's Witnesses.  I began to unknowingly remove myself from my faith. The first scripture my father had me memorize was \bibleverse{IJohn}(5:19). I feel that I am guilty alone for my division from God but in this case I also feel my environment was a helpful contributer. \bibleverse{IJohn}(5:19)~\cite{niv} gives a pretty clear warning:
\begin{verse}
We know that we are children of God,† and that the whole world is under the control of the evil one.
\end{verse}
A pretty chilling thought. At this point in my life I was no longer under the protective umbrella of my parents and I began questioning my beliefs.
\end{doublespace}
\section{Billabong}
\begin{doublespace}
After years of living in Maine and half heartedly remaining there I was tired of the partying lifestyle I had been submerged in. I moved to the Outer Banks of North Carolina and left the college town of Orono in Maine. I had not been applying myself in Orono anyway so this move was a wise one. Incredibly, with all the horrible choices I had made in the past few years my move to the Carolinas was a good one. I did not begin seeking God again but this move took me out of the bad lifestyle I was living and got me back on my feet. While my river of life was dried up there was still a Billabong of water inside of me which retained hope. Days when I would go surfing I would appreciate his beautiful creations. Later I joined the Navy which did not enhance my spiritual relationship with God but helped me form a career that would help me pick up the pieces of bad choices in the past. Once I had formed a stable foundation again I was free to consider bigger things in life beyond food and shelter. I began reading book while underway such as the Quran and The Sagas of the Icelanders~\cite{Viking}. While these books are very different they were a product of my spiritual search beginning to emit sparks. The Quran a wonderful and holy book to Muslims gave me incite to the culture I was surrounded by while in the Persian Gulf. The Sagas of Icelanders for the most part is an enjoyable book full of stories from early Norsemen. This book forced me to start wondering again what it must have been like for men who had never even heard of Jesus or YAHWEH. I still could not bring myself to open an Old or New Testament even when the chaplains would hand them out constantly. To some degree I was terrified to look and therefore admit to my lifestyle. 
\end{doublespace}
    \section{River Mouth}
\clearpage
  % bib stuff
    \nocite{*}
    \addtocontents{toc}{\protect\vspace{\beforebibskip}}
    \addcontentsline{toc}{section}{\refname}    
    \bibliographystyle{plainnat}
    \bibliography{cite}
\end{document}